\documentclass[a4paper, 12pt]{article}
\usepackage[utf8]{inputenc}
\usepackage[T1]{fontenc}
\usepackage[italian]{babel}
\usepackage{a4wide}
\usepackage{graphicx}
\usepackage[table]{xcolor}
\usepackage{geometry}
\geometry{a4paper,top=1.25cm,bottom=4cm,left=3cm,right=3cm}
\graphicspath{{pics/}}

%TITOLO
%%%%%%%%%%%%%%%%%%%%%%%%%%%%%%%%%%%%%%%%%%%%%%%%%%%%%%%%%%%%%%%%%%%%%
\newcommand{\Title}{Crittografia classica e quantistica}
\newcommand{\TITLE}{CRITTOGRAFIA CLASSICA E QUANTISTICA}
%%%%%%%%%%%%%%%%%%%%%%%%%%%%%%%%%%%%%%%%%%%%%%%%%%%%%%%%%%%%%%%%%%%%%

%VERSIONE
%%%%%%%%%%%%%%%%%%%%%%%%%%%%%%%%%%%%%%%%%%%%%%%%%%%%%%%%%%%%%%%%%%%%%
\newcommand{\theversion}{ X.X.X }
%%%%%%%%%%%%%%%%%%%%%%%%%%%%%%%%%%%%%%%%%%%%%%%%%%%%%%%%%%%%%%%%%%%%%

%DATA
%%%%%%%%%%%%%%%%%%%%%%%%%%%%%%%%%%%%%%%%%%%%%%%%%%%%%%%%%%%%%%%%%%%%%
\newcommand{\thedate}{ G Mese Anno }
%%%%%%%%%%%%%%%%%%%%%%%%%%%%%%%%%%%%%%%%%%%%%%%%%%%%%%%%%%%%%%%%%%%%%

%VERIFICATORI
%%%%%%%%%%%%%%%%%%%%%%%%%%%%%%%%%%%%%%%%%%%%%%%%%%%%%%%%%%%%%%%%%%%%%
\newcommand{\verif}{Primo Verificatore, Secondo Verificatore}
%%%%%%%%%%%%%%%%%%%%%%%%%%%%%%%%%%%%%%%%%%%%%%%%%%%%%%%%%%%%%%%%%%%%%

%RESPONSABILE
%%%%%%%%%%%%%%%%%%%%%%%%%%%%%%%%%%%%%%%%%%%%%%%%%%%%%%%%%%%%%%%%%%%%%
\newcommand{\resp}{Unico Responsabile}
%%%%%%%%%%%%%%%%%%%%%%%%%%%%%%%%%%%%%%%%%%%%%%%%%%%%%%%%%%%%%%%%%%%%%

%DESCRIZIONE
%%%%%%%%%%%%%%%%%%%%%%%%%%%%%%%%%%%%%%%%%%%%%%%%%%%%%%%%%%%%%%%%%%%%%
\newcommand{\descript}{\input{../includes/description}}
%%%%%%%%%%%%%%%%%%%%%%%%%%%%%%%%%%%%%%%%%%%%%%%%%%%%%%%%%%%%%%%%%%%%%

\usepackage{ifthen}
\usepackage{ifpdf}
\ifpdf
\usepackage[pdftex]{hyperref}
\else
\usepackage{hyperref}
\fi
\usepackage{color}
\hypersetup{%
colorlinks=true,
linkcolor=black,
citecolor=black,
urlcolor=blue}
\usepackage[nonumberlist]{glossaries}
\usepackage{afterpage}
\setcounter{secnumdepth}{5}
\setcounter{tocdepth}{5}
\usepackage{subfig}
\usepackage{tikz}
\usepackage{tabularx}
\usetikzlibrary{shapes,arrows}
\usepackage{pgfplots}
\pgfplotsset{compat=newest}
\pgfplotsset{plot coordinates/math parser=false}
\newlength\figureheight
\newlength\figurewidth
\pgfkeys{/pgf/number format/.cd,
set decimal separator={,\!},
1000 sep={\,},
}
\renewcommand{\baselinestretch}{1.05}

%FANCYPAGESTYLE
%%%%%%%%%%%%%%%%%%%%%%%%%%%%%%%%%%%%%%%%%%%%%%%%%%%%%%%%%%%%%%%%%%%%%
\usepackage{fancyhdr}
\pagestyle{fancy}
\fancyfoot[R]{\thepage}
\fancyfoot[L]{\Title{}}
\fancyfoot[C]{}
\fancyhead[R]{\includegraphics[width=0.12\textwidth]{unipd.png}}
\fancyhead[L]{Daniele Giachetto}
\setlength{\headheight}{60pt}
%%%%%%%%%%%%%%%%%%%%%%%%%%%%%%%%%%%%%%%%%%%%%%%%%%%%%%%%%%%%%%%%%%%%%

\let\headruleORIG\headrule
\renewcommand{\headrule}{\color{black} \headruleORIG}
\renewcommand{\headrulewidth}{1.0pt}
\usepackage{colortbl}
\arrayrulecolor{black}

\let\footruleORIG\footrule
\renewcommand{\footrule}{\color{black} \footruleORIG}
\renewcommand{\footrulewidth}{1.0pt}

\fancypagestyle{plain}{
  \fancyhead{}
  \fancyfoot[C]{\thepage}
  \renewcommand{\headrulewidth}{0pt}
}

\usepackage{amsthm}
\usepackage{amssymb,amsmath}
\usepackage{array}
\usepackage{bm}
\usepackage{multirow}
\usepackage[footnote]{acronym}


\newcommand\blankpage{%
    \null
    \newpage}

\parskip=5pt

\begin{document}
%PRIMA PAGINA
%%%%%%%%%%%%%%%%%%%%%%%%%%%%%%%%%%%%%%%%%%%%%%%%%%%%%%%%%%%%%%%%%%%%%
\begin{titlepage}
\begin{center}
\includegraphics[width=0.5\textwidth]{unipd.png}\\

{\large \TITLE{} \\ a.a. 2020/2021}\\[0.5cm]

\begin{tabular}{ll}

\end{tabular}
\\[1.cm]

\rule{\linewidth}{0.5mm} \\[0.4cm]
{ \huge \bfseries \TITLE{} \\[0.4cm] } %TITOLO

\rule{\linewidth}{0.5mm} \\[1.cm]
\noindent

\vfill
{\large \textit{1201145} - DANIELE GIACHETTO} %BOTTOM
\end{center}
\end{titlepage}

%%%%%%%%%%%%%%%%%%%%%%%%%%%%%%%%%%%%%%%%%%%%%%%%%%%%%%%%%%%%%%%%%%%%%
\tableofcontents %crea indice
\clearpage %pagina nuova

\pagestyle{fancy}
\section{La segretezza nella comunicazione}
 
La crittografia è la branca della scienza che tratta i metodi per rendere un messaggio inintelligibile per chiunque non sia in possesso di determinati strumenti.\newline
Lo scopo di questo elaborato è quello di esporre la storia e l'evoluzione di questa disciplina, partendo dai suoi primi usi documentati fino ad arrivare al suo presente, in continuo e costante sviluppo.

\subsection{Il passato della crittografia}


\section{Il presente}

\section{Il futuro}


\clearpage

\section{Fonti}

{\large \textbf{Bibliografia}\par}
\begin{itemize}
    \item Eva Filoramo, Alberto Giovannini e Claudia Pasquero \textit{Alla scoperta della crittografia quantistica}, prima edizione settembre 2006.
\end{itemize}

{\large \textbf{Sitografia}\par}
\begin{itemize}
    \item Wikipedia, \textit{Crittografia}, 
    \item[] \href{https://lmo.wikipedia.org/wiki/Crittografia}{https://lmo.wikipedia.org/wiki/Crittografia}
\end{itemize}

\end{document}